\documentclass[%
	11pt,% bigger font
	a4paper,% paper size
	pagesize,% set pagesize in PDF
	oneside,% like a book
	listof=totoc,%add list of figures to toc
	draft% TODO remove this!
]{scrbook}

%---------------------------------------
%---------PACKAGES----------------------
%---------------------------------------

% font config
\usepackage{libertine}
\usepackage[T1]{fontenc}

% microtype
\usepackage[activate={true,nocompatibility},final,tracking=true,kerning=true,spacing=true,factor=1100,stretch=10,shrink=10]{microtype}
% activate={true,nocompatibility} - activate protrusion and expansion
% final - enable microtype; use "draft" to disable
% tracking=true, kerning=true, spacing=true - activate these techniques
% factor=1100 - add 10% to the protrusion amount (default is 1000)
% stretch=10, shrink=10 - reduce stretchability/shrinkability (default is 20/20)

% links
\usepackage{hyperref}

% math
\usepackage{amsmath}

% graphics
\usepackage{tikz}
\usepackage{tikz-3dplot}

% glossar
% dep: hyperref
\usepackage[
	toc%add to toc
]{glossaries}

% TODO remove this!
% debug packages
\usepackage{blindtext}


%---------------------------------------
%---------SETTINGS----------------------
%---------------------------------------

% general informaion
\title{Some funny words}
\author{Marco Neumann}

% color
\definecolor{colcontrast}{RGB}{255,0,0}

% links
\hypersetup{
	colorlinks,% colored text instead of borders
	linkcolor=black,%  black links
	final% also work in draft mode, TODO remove this!
}

% tikz
\usetikzlibrary{arrows}
\tikzset{
	>=stealth'
}


%---------------------------------------
%---------GLOSSARY----------------------
%---------------------------------------

\newglossaryentry{random number generator}
{
	name={Random Number Generator},
	description={Generates random numbers}
}

\newacronym{rng}{RNG}{\glslink{random number generator}{Random Number Generator}}

\makeglossaries% build glossary
\glsunsetall% fix acronyms


\begin{document}

\frontmatter
\maketitle
\tableofcontents

\mainmatter
\chapter{How to beat the system}
Consider the following, constructed dataset: Generate some random points $(x, y, z) \in \left[0, 1\right)^3$, that satisfy the following constraint:
\begin{equation}\label{eq:beat}
	(x + y + z) \bmod 1 = 0
\end{equation}
If you sample enough points using a good uniform pseudo \gls{rng}, you will get the following destributions:
\begin{figure}
	\begin{tikzpicture}[scale = 5.0]
		\begin{scope}
			\foreach \name [count=\d from 0] in {x,y,z} {
				\begin{scope}[yshift=-\d*12]
					\draw (-0.1,0) node {\name};
					\begin{scope}[scale=0.8]
						\draw[->] (0,0) -- (1,0);
						\foreach \x in {0,...,9} {
							\fill[colcontrast] (\x/10,0) circle(0.2pt);
						}
					\end{scope}
				\end{scope}
			}
		\end{scope}
		\begin{scope}[shift={(0,-1.7)}]
			\foreach \name [count=\d from 0] in {{x,y},{y,z},{z,x}} {
				\begin{scope}[xshift=\d*25]
					\draw (-0.1,-0.1) node {\name};
					\begin{scope}[scale=0.4]
						\draw[->] (0,0) -- (1,0);
						\draw[->] (0,0) -- (0,1);
						\foreach \x in {0,...,9} {
							\foreach \y in {0,...,9} {
								\fill[colcontrast] (\x/10,\y/10) circle(0.2pt);
							}
						}
					\end{scope}
				\end{scope}
			}
		\end{scope}
		\begin{scope}[shift={(1.2,-0.9)}]
			\draw (-0.1,-0.1) node {x,y,z};
			\begin{scope}[scale=0.8]
				\draw[->] (0,0,0) -- (1,0,0);
				\draw[->] (0,0,0) -- (0,1,0);
				\draw[->] (0,0,0) -- (0,0,-1);
				\foreach \x in {0,...,9} {
					\foreach \y in {0,...,9} {
						\pgfmathtruncatemacro{\z}{mod(\x+\y,10)}
						\fill[colcontrast] (\x/10,\y/10,-\z/10) circle(0.3pt-0.1pt*\z/10);
					}
				}
			\end{scope}
		\end{scope}
	\end{tikzpicture}
	\caption{Situation described in \ref{eq:beat}}
	\label{fig:beat}
\end{figure}

\appendix

\backmatter
\listoffigures
\printglossaries

\end{document}

