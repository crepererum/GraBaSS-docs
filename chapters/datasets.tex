\chapter{Datasets}
High dimensional datasets are not a very common research topics. In this chapter I will describe some data sources and possible applications of the described algorithm. I also show some interpretations of subspaces in different datasets and some failed approaches of finding them. To get, isolate and preprocess the data I've used some scripts. Because I think open research should not only contain open publications but also open data sources, you can find the scripts and some notes about them online.\footnote{\url{https://github.com/crepererum/GSD}} Should you have some questions, find a bug or want to contribute, feel free to use the issue tracker or file a pull request.

\section{Audio Spectrum}
An audio spectrum can characterize the underlying material. I analyzed the songs from the following collections:
\begin{itemize}
	\item Daft Punk -- Random Access Memories (\circled{P}2013 by Daft Life, Columbia), \num{13} songs
	\item Mayday 2013 -- Never Stop Raving (\circled{C}2013 by Contor Records GmbH), \num{60} songs
\end{itemize}
I converted the audio to a sample rate of \SI{44100}{\hertz}. After it, I splitted the audio files into parts containing \num{4096} parts, applied the Hamming window and passed it to a FFT. The FFT produces a output vector with \num{4096} values, but only \num{2048} values are usable, because the spectrum should be mirrored. The values are normalized and converted to \si{\decibel}. Then I calculated the mean of \num{32} parts to supress noise, which is a common audio effect in modern music, but may also effect classical records. This mean vector has a length of \num{2048} and is written to a file. So, the mean of \num{32} parts is one data element. Figure~\ref{fig:audio} shows some of the results.
\begin{figure}
	\begin{tikzpicture}
	\def\xmin{0}
	\def\xmax{2.048}
	\def\xstep{0.2}
	\def\ymin{-120}
	\def\ymax{0}
	\def\ystep{5}

	\begin{scope}[xscale=5,yscale=0.04]
		\draw[style=help lines, ystep=20, xstep=0.256] (\xmin,\ymin) grid(\xmax,\ymax);

		\foreach \x in {\xmin,0.256,...,\xmax} {
			\node at (\x,\ymin) [below]{\num[round-precision=3,round-mode=places]{\x}};
		}
		\foreach \y in {\ymin,-100,...,\ymax} {
			\node at (\xmin,\y) [left]{\num{\y}};
		}

		\draw[colcontrast] plot[smooth] file{data/audio1.dat} node[right]{example 1};
		\draw[colcontrast2] plot[smooth] file{data/audio2.dat} node[right]{example 2};

		\draw[axis] (\xmin,\ymin) -- (\xmax,\ymin) node[right]{\si{\kilo\hertz}};
		\draw[axis] (\xmin,\ymin) -- (\xmin,\ymax) node[above]{\si{\decibel}};
	\end{scope}
\end{tikzpicture}


	\caption{Audio spectrum}
	\label{fig:audio}
\end{figure}

The subspace analyzis performance well but gives boring results. Meanly the first and last bins of the spectrum forms one subsapce and the middle part describes another.

\section{Drug Database}
Another source of high dimensional data is the list of incredients of drugs. A given set of drugs described by a list of their incredients is processed as followed: Build a list of all incredients in all drugs and bring them into a fixed order. Every incredient forms one dimension. Then build a binary vector for every drug in the database where $1$ describes that a incredient is contained in the drug and $0$ if this is not the case.
I hoped to find some public drug data bases containing drugs registered in Germany or in the EU. But this was not the case. Either they were only usable via a very limited interface or it would cost me a enorm amount of money to buy a license. Since the open data movement in USA has a very long tradition, it was easy to find an american drug database, that is easy to parse and provides a huge data collection.\footnote{\url{http://dailymed.nlm.nih.gov/dailymed/downloadLabels.cfm}} I provide a parser and converter for this kind of data.

