\chapter{Designing a high perfomance algorithm}\label{chap:algodesign}
To handle current and future big datasets a special approach is required. In this chapter, I will design a high perfomance algorithm that is able to handle this task. The main goal is a method that does not beat the comlexity of $\mathcal{O} ( \card{D} \cdot \card{N} \cdot \log{\card{N}} + \card{D}^2 \cdot \card{N} )$. This complexity is a good upper bound. It combines the bounds of two different task. The first one is the preprocessing and structure detection of all objects in all dimensions. $\mathcal{O} ( \card{D} \cdot \card{N} )$ would be enough to normalize the data, but $\mathcal{O} ( \card{D} \cdot \card{N} \cdot \log{\card{N}} )$ allowes some better methods like sorting. The second part is the detection of relations between all dimensions, which could be done in $\mathcal{O} ( \card{D}^2 )$. Because reducing the dimenions to a fixed set of features does honor a non fixed number of elements, a good method needs $\mathcal{O} ( \card{D}^2 \cdot \card{N} )$.

\section{From similarity to subspace}
Based on the combined similarity it is possible to calculate subspaces. It is possible to calculate distances between all pairs of dimensions, because of the nature of subspaces. So it is important to find groups in this $\card{D}^2$ relations highly efficient.

Next, I want to discuss how subspaces are formed. Two dimenions should be contained in the subspaces, when they are similar. This can be expressed by a threshold $p_t \in [0,1]$, which the combined similarity should have. By stripping down the similarity matrix to a binary matrix, the dimenions form a graph. This speeds up the calculation, because it allows fast set operations and memory effective management. Because the most data sets have very different dimensions, the number of edges should be low.

Another important attribute of subspaces is that all contained dimenions are similar to each other. This attribute is already described as clique. Because the graph is dense it has a low degeneracy $d \in \set{N}_0$. So, the cliques can be calucated in $\mathcal{O}(d \cdot (\card{D} - d) \cdot 3^{\frac{d}{3}})$ by using \cite{listingCliques}. Because the cliques forms the subspaces, this is the complete algorithm for converting the combined similarity to subspaces.

\section{Optimize graph structure}
A core problem of strict clique searchers is the heavy fragmentation if only one edge missing. Figure~\ref{subfig:opt_graph1} shows an example where the nodes semi clique $\left\{1,2,3,4\right\}$ is splitted into two cliques because of the absence of the edge $(1,3)$. For some applications it could be helpful to add this edge to the graph and join the two cliques into one. The first method to do it is the utilyzation a distance graph:
\begin{envdef}[Strict Distance Graph]
	Given a graph $G=(V,E)$ and a distance $n \in \set{N}_0$, the strict distance graph $G^n=(V,E^n)$ is give by:
	\begin{align}
		E^0 &= \left\{ (v,v) \in V^2 \right\}\\
		E^i &= \left\{ (v_0,v_2) \in V^2 \mid \exists v_1 \in V : (v_0,v_1) \in E^{(i-1)}, (v_1,v_2) \in E \right\}, \quad \forall i>0
	\end{align}
\end{envdef}
\begin{figure}
	\caption{Different subspaces from different graph preprocessing}
	\label{fig:opt_graph}
	\centering
	\subfloat[\label{subfig:opt_graph1}Too many subspaces]{\begin{tikzpicture}
	\node[graphnode] (1) {1};
	\node[graphnode] (2) [above right of=1] {2};
	\node[graphnode] (3) [below right of=2] {3};
	\node[graphnode] (4) [below right of=1] {4};
	\node[graphnode] (5) [below=30pt of 4] {5};
	\node[graphnode] (6) [below left of=5] {6};
	\node[graphnode] (7) [below right of=5] {7};

	\path[graphedge, color=colcontrast]
		(1) edge (2)
			edge (4)
	;

	\path[graphedge, color=colcontrast2]
		(3) edge (2)
			edge (4)
	;

	\path[graphedge, color=colcontrast, dash pattern=on 4pt off 4pt]
		(2) edge (4)
	;

	\path[graphedge, color=colcontrast2, dash pattern=on 4pt off 4pt, dash phase=4pt]
		(2) edge (4)
	;

	\path[graphedge, color=colcontrast3]
		(4) edge (5)
	;

	\path[graphedge, color=colcontrast4]
		(5) edge (6)
			edge (7)
		(6) edge (7)
	;
\end{tikzpicture}

}
	\hfill
	\subfloat[\label{subfig:opt_graph2}Simple graph distance]{\begin{tikzpicture}
	\node[graphnode] (1) {1};
	\node[graphnode] (2) [above right of=1] {2};
	\node[graphnode] (3) [below right of=2] {3};
	\node[graphnode] (4) [below right of=1] {4};
	\node[graphnode] (5) [below=30pt of 4] {5};
	\node[graphnode] (6) [below left of=5] {6};
	\node[graphnode] (7) [below right of=5] {7};

	\path[graphedge, color=colcontrast]
		(2) edge (1)
			edge (3)
			edge (4)
		(4) edge (1)
			edge (3)
	;

	\path[graphedge, color=colcontrast2]
		(5) edge (6)
			edge (7)
		(6) edge (7)
	;

	\path[graphedge, color=colcontrast, dash pattern=on 4pt off 4pt]
		(4) edge (5)
	;

	\path[graphedge, color=colcontrast2, dash pattern=on 4pt off 4pt, dash phase=4pt]
		(4) edge (5)
	;

	\path[graphedge, color=colcontrast, dotted]
		(1) edge (3)
		(5) edge (1)
			edge[bend right=90,looseness=1.5] (2)
			edge (3)
	;

	\path[graphedge, color=colcontrast2, dotted]
		(4) edge (6)
			edge (7)
	;
\end{tikzpicture}

}
	\hfill
	\subfloat[\label{subfig:opt_graph3}Filtered distance with $\alpha=\frac23$]{\begin{tikzpicture}
	\node[graphnode] (1) {1};
	\node[graphnode] (2) [above right of=1] {2};
	\node[graphnode] (3) [below right of=2] {3};
	\node[graphnode] (4) [below right of=1] {4};
	\node[graphnode] (5) [below=30pt of 4] {5};
	\node[graphnode] (6) [below left of=5] {6};
	\node[graphnode] (7) [below right of=5] {7};

	\path[graphedge]
		(2) edge (1)
			edge (3)
		(4) edge (1)
			edge (2)
			edge (3)
			edge (5)
		(5) edge (6)
			edge (7)
		(6) edge (7)
	;

	\path[graphedge, dotted]
		(1) edge[<->] (3)
		(5) edge[->] (1)
			edge[->] (3)
		(4) edge[->] (6)
			edge[->] (7)
	;
\end{tikzpicture}

}
	\hfill
	\subfloat[\label{subfig:opt_graph4}Bidirectional filtered distance with $\alpha=\frac23$]{\begin{tikzpicture}
	\node[graphnode] (1) {1};
	\node[graphnode] (2) [above right of=1] {2};
	\node[graphnode] (3) [below right of=2] {3};
	\node[graphnode] (4) [below right of=1] {4};
	\node[graphnode] (5) [below=30pt of 4] {5};
	\node[graphnode] (6) [below left of=5] {6};
	\node[graphnode] (7) [below right of=5] {7};

	\path[graphedge, color=colcontrast]
		(2) edge (1)
			edge (3)
			edge (4)
		(4) edge (1)
			edge (3)
	;

	\path[graphedge, color=colcontrast, dotted]
		(1) edge (3)
	;

	\path[graphedge, color=colcontrast2]
		(4) edge (5)
	;

	\path[graphedge, color=colcontrast3]
		(5) edge (6)
			edge (7)
		(6) edge (7)
	;
\end{tikzpicture}

}
\end{figure}
Because it does not make any sense to forget about the given edges when calcuating higher distances, the definition of the joined distance graph ist more useful:
\begin{envdef}[Joined Distance Graph]
	Given a graph $G=(V,E)$ and a distance $n \in \set{N}_+$, the joined distance graph $G^{\overline{n}}=(V,E^{\overline{n}})$ is definied as followed:
	\begin{equation}
		E^{\overline{n}} = \bigcup_{i=1}^n E^i
	\end{equation}
\end{envdef}
Using the joined distance graph fixes the problem, but does not produce the desired result. As seen in Figure~\ref{subfig:opt_graph2}, it attaches to many new vertices to the cliques and make the results unusable. The reason is that it does not count the strenght of the connection in the exisiting graph and so it creates some edges to weakly connected vertices. A messure for the connection rate between two vertices can be calculated by the following equation:
\begin{envdef}[Connection Rate]
	Given a graph $G=(V,E)$ and two vertices $v_1,v_2 \in V$, the connection rate $r_G(v_1,v_2) \in \left[0,1\right]$ can be calcuated by the relation of connected to all neighbors of $v_1$:
	\begin{equation}
		r_G(v_1,v_2) = \frac{
			\card{ \left\{ v \in V \mid (v_1, v) \in E \right\} \cap \left\{ v \in V \mid (v,v_2) \in E \right\} }
		}{
			\card{ \left\{ v \in V \mid (v_1, v) \in E \right\} }
		}
	\end{equation}
\end{envdef}
Please notice that the connection rate is not symmetric. To honor the connection rate of the vertices, I use a filtered strict distance graph:
\begin{envdef}[Filtered Strict Distance Graph]
	Given a graph $G=(V,E)$, a distance $n \in \set{N}_+$ and a filter rate $\alpha \in \left[0,1\right]$, the filtered strict distance graph $G_\alpha^n=(V,E_\alpha^n)$ is definined as followed:
	\begin{equation}
		E_\alpha^n = \left\{ (v_1,v_2) \in E^n \mid r_{G^{n-1}}(v_1,v_2) \geq \alpha \right\}
	\end{equation}
\end{envdef}
The joined filtered graph, that considers all distances up to a given value, is definied analog:
\begin{envdef}[Filtered Joined Distance Graph]
	Given a graph $G=(V,E)$a distance $n \in \set{N}_+$ and a filter rate $\alpha \in \left[0,1\right]$, the filtered joined distance graph $G_\alpha^{\overline{n}}=(V,E_\alpha^{\overline{n}})$ is definied as followed:
	\begin{equation}
		E_\alpha^{\overline{n}} = \bigcup_{i=1}^n E_\alpha^i
	\end{equation}
\end{envdef}
The filtered joined distance graph is not bidirectional. This leads the problem that there are unidirectional edges that are not usable for our result. As Figure~\ref{subfig:opt_graph3} shows, the weakly connected nodes are connected to the semiclique whereas the semiclique is not connected to the satallites. The inter semiclique edges are bidirectional. Based on this fact, a bidirectional graph can be constructed:
\begin{envdef}[Bidirectional Joined Filtered Distance Graph]
	Given a graph $G=(V,E)$, a distance $n \in \set{N}_+$ and a filter rate $\alpha \in \left[0,1\right]$, the bidirectional joined filtered graph $\tilde{G}_\alpha^{\overline{n}}=(V,\tilde{E}_\alpha^{\overline{n}})$ is defined by the subset of the bidirectional edges of the joined filtered graph:
	\begin{equation}
		\tilde{E}_\alpha^{\overline{n}} = \left\{ (v_1,v_2) \in V^2 \mid (v_1,v_2),(v_2,v_1) \in E_\alpha^{\overline{n}} \right\}
	\end{equation}
\end{envdef}
As shown on Figure~\ref{subfig:opt_graph4} this method successfully adds inter semiclique edges to join them to cliques without creating senseless connections or resulting in informaion loosage.

\begin{algorithm}
	\KwData{Vertex $v$}
	\KwData{Graph $G$}
	\KwData{Threshold $t$}
	\KwResult{New neighbors $N$}

	\SetKwFunction{CalcConnectionRate}{calcConnectionRate}
	\SetKwFunction{GetNeighbors}{getNeighbors}

	\Begin{
		\tcc{Search all candidates}
		$C$ $\leftarrow$ $\{\}$\;
		\For{$w \in \GetNeighbors{$v$}$}{
			$C$ $\leftarrow$ $C \cup \GetNeighbors{$w$}$\;
		}

		\BlankLine
		\tcc{Check all candidates}
		$N$ $\leftarrow$ $\{\}$\;
		\For{$c \in C$}{
			\If{$\CalcConnectionRate{$v$, $c$} \geq t \wedge \CalcConnectionRate{$c$, $v$} \geq t$}{
				$N$ $\leftarrow$ $N \cup \{c\}$\;
			}
		}
	}
	\caption{extendNeighbors}
\end{algorithm}

\begin{algorithm}
	\KwData{Graph $G$}
	\KwData{Distance $d$}
	\KwData{Threshold $t$}
	\KwResult{Refined Graph $G_r$}

	\SetKwFunction{ExtendNeighbors}{extendNeighbors}

	\Begin{
		$G_r$ $\leftarrow$ $G$\;
		\For{i=2 \emph{\KwTo} $d$}{
			$G_n$ $\leftarrow$ $()$\;
			\For{$v \in G_r$}{
				$G_n$ $\leftarrow$ $G_n \concat \ExtendNeighbors{$v$, $G_r$, $t$}$\;
			}
			$G_r$ $\leftarrow$ $G_n$\;
		}
	}
	\caption{refineGraph}
\end{algorithm}

To calculate the bidirectional joined filtered distance graph, I assume that the algorithm gets the distance $n \in \set{N}_+$ as a fixed, input size independet parameter. To graph will generated from the graph with the distance $n-1$ where a distance of $1$ is the input graph. Every of this rounds contains two steps. Step one is the calculation of the connection rates, which can be done in $\mathcal{O} ( \card{V}^2 )$. This assumes that data structures are used that can answert the question ``Is vertix $v_1$ connceted to $v_2$?'' in $\mathcal{O} ( 1 )$. The second calculation step is the construction of the new neighborhood for every node. Assuming that you use $\mathcal{O} ( 1 )$ set operations, this can be done for all vertices in $\mathcal{O} ( \card{V}^2 )$. This results in a total cost per round of $\mathcal{O} ( \card{V}^2 )$ and a total algorithm cost of $\mathcal{O} ( n \cdot \card{V}^2 ) = \mathcal{O} ( \card{V}^2 )$. Following the advice to choosing the distance parameter $n$ independet of input size, this graph transformation will approximatly not slow down the entire algorithm.

\section{Algorithm Summary}
\begin{algorithm}
	\KwData{Dimension $d$}
	\KwData{Binsize $s$}
	\KwResult{Bins $B$}

	\SetKwData{Undef}{undef}

	\SetKwFunction{Sort}{sort}

	\Begin{
		$B$ $\leftarrow$ $()$\;
		$d_s$ $\leftarrow$ \Sort{$d$}\;
		$x$ $\leftarrow$ \Undef\;
		$b$ $\leftarrow$ $()$\;
		\For{$x \in d_s$}{
			\If{$|B| \geq s \wedge x \neq l$}{
				$B$ $\leftarrow$ $B \concat b$\;
				$b$ $\leftarrow$ $()$\;
			}
			$b$ $\leftarrow$ $B \concat x$\;
			$l$ $\leftarrow$ $x$\;
		}
		\If{$|b| \neq 0$}{
			$B$ $\leftarrow$ $B \concat b$\;
		}
	}
	\caption{buildBins}
\end{algorithm}

\begin{algorithm}
	\KwData{Dataset $D$}
	\KwData{Threshold $p_t$}
	\KwData{Refine threshold $p_r$}
	\KwResult{Subspaces $S$}

	\SetKwFunction{BuildBins}{buildBins}
	\SetKwFunction{CalcMeanValues}{calcMeanValues}
	\SetKwFunction{CalcStddevValues}{calcStddevValues}
	\SetKwFunction{CalcVarValues}{calcVarValues}
	\SetKwFunction{CoVar}{coVar}
	\SetKwFunction{DimSimilarity}{dimSimilarity}
	\SetKwFunction{RefineGraph}{refineGraph}
	\SetKwFunction{SearchCliques}{searchCliques}

	\Begin{
		\tcc{Pre calculate aggregates}
		$D_m$ $\leftarrow$ \CalcMeanValues{$D$}\;
		$D_v$ $\leftarrow$ \CalcVarValues{$D$, $D_m$}\;
		$D_s$ $\leftarrow$ \CalcStddevValues{$D_v$}\;

		\BlankLine
		\tcc{Build grid}
		$D_g$ $\leftarrow$ $()$\;
		\For{$d \in D$}{
			$D_g$ $\leftarrow$ $D_g \concat$ \BuildBins{$d$, $\sqrt{|D|}$}\;
		}

		\BlankLine
		\tcc{Build graph}
		\For{$(d_1,m_1,s_1,g_1) \in (D,D_m,D_s,D_g)$}{
			$n$ $\leftarrow$ $\{\}$\;
			\For{$(d_2,m_2,s_2,g_2) \in (D,D_m,D_s,D_g)\setminus\{(d_1,m_1,s_1,g_1)\}$}{
				$x_1$ $\leftarrow$ $\frac{\CoVar{$d_1$, $d_2$, $m_1$, $m_2$}}{s_1 \cdot s_2}$\;
				$x_2$ $\leftarrow$ \DimSimilarity{$g_1$, $g_2$}\;
				\If{$x_1 \cdot x_2 \geq p_t$}{
					$n$ $\leftarrow$ $n \cup \{d_2\}$\;
				}
			}
			$G$ $\leftarrow$ $G \concat n$\;
		}

		\BlankLine
		\tcc{Refine graph}
		$G_r$ $\leftarrow$ \RefineGraph{$G$, $p_n$, $p_r$}\;

		\BlankLine
		\tcc{Search cliques (which are our subspaces)}
		$S$ $\leftarrow$ \SearchCliques{$G_r$}\;
	}
	\caption{calcSubspaces}
\end{algorithm}

